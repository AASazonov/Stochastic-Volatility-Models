\chapter{Implementation of the Methods}
    \section{General Problems and \texttt{mc\_price} Function}
        The error of discretization consists of two parts: the discretization error itself 
        (the one coming from the transition from an stochastic differential equation to the 
        stochastic difference equation) and the Monte-Carlo error (see Section \ref{Monte-Carlo:statistical}).
        We controlled the Monte-Carlo error with the following method:
        \begin{algorithm}
            \caption{Outer loop of the Monte-Carlo method (\texttt{mc\_price})}
            \label{alg:outer_loop}
            \begin{algorithmic}
                \State {\bf Inputs}: \texttt{payoff}, \texttt{simulate}, market parameters, Heston parameters.
                \State prices = $\left[ \ \right]$
                \While{{len(prices\_confidence\_interval) > desired precision} \textbf{and} iter < MAX\_ITER}
                    \State paths = simulate(batch\_size)
                    \State prices $\gets$ payoff(paths)
                    \State prices\_confidence\_interval = confidence\_interval(prices)
                \EndWhile
                \State return mean(prices)
            \end{algorithmic}
        \end{algorithm}

    \section{Discretisation Schemes}

        \begin{algorithm}[htbp]
            \caption{Euler-Maruyama scheme (\texttt{simulate\_heston\_euler})}
            \label{alg:euler}
            \begin{algorithmic}
                \State {\bf Inputs}: $S_0$, $V_0^*$, $r$, $\kappa$, $\theta$, $\sigma$, $\rho$, $T$, $N$
                \State Set $X_0 = \log S_0$
                \State Set $V_0 = V_0^*$
                \State Generate a 2-dimensional Gaussian noise $Z_i = \left[\begin{aligned}
                    Z_i^1\\
                    Z_i^2               
                \end{aligned}\right] \sim \cN\left(0,\left[\begin{array}{cc}
                    1 & 0\\
                    0 & 1
                \end{array}\right]\right)$
                \For{$i = 1$ to $N$}
                    \State $X_{i+1}  = X_i + (\mu - 0.5 V_i^+)h_i + \sqrt{V_i^+} X_i \sqrt{h_i} Z_i^1$
                    \State $ V_{i+1}  = V_i + \left(\delta^2 - 2\beta V_i^+\right) h_i + \sigma \sqrt{V_i^+} \left(\rho Z_i^1 + \sqrt{1-\rho^2}Z_i^2\right)$
                \EndFor
                \State return $S_N = \exp X_N$
            \end{algorithmic}
        \end{algorithm}

        \begin{algorithm}[htbp]
            \caption{Quadratic-Exponential scheme (\texttt{simulate\_heston\_andersen\_qe})}
            \label{alg:andersen_qe}
            \begin{algorithmic}
                \State {\bf Inputs}: 
                \State Set $X_0 = \log S_0$
                \State Set $V_0 = V_0^*$
                \State Set some $\psi_c \in [1, 2]$
                \For{$i = 1$ to $N$}
                    \State Compute $m$ and $s^2$ given $V_{i-1}$.
                    \State Compute the coefficient of variation $\psi = \frac{s^2}{m^2}$
                    \State Generate a random number $U_V \sim U(0,1)$
                    \If {$\psi \leq \psi_c$}
                        \State Compute $a$ and $b$ (eq. \eqref{eq:quadratic:a} -- \eqref{eq:quadratic:b})
                        \State Compute $Z_V \sim \cN(0,1)$ using the Smirnov transform applied to $U_V$
                        \State Compute $V_{i+1} = a\left(b+Z_V\right)^2$
                    \Else
                        \State
                    \EndIf
                \EndFor
            \end{algorithmic}
        \end{algorithm}

        \begin{algorithm}[htbp]
            \caption{Truncated Gaussian scheme (\texttt{simulate\_heston\_andersen\_tg})}
            \label{alg:andersen_tg}
            \begin{algorithmic}
                \State {\bf Inputs}: 
            \end{algorithmic}
        \end{algorithm}

        %\begin{algorithm}[htbp]\caption{Exact scheme (\texttt{simulate\_heston\_exact})}\label{alg:exact}
            %\begin{algorithmic}
            %    \State {\bf Inputs}: 
            %\end{algorithmic}
        %\end{algorithm}


\chapter{Comparison of the Methods}
    We shall compare the described methods for the European call option prices due to the fact that we have a closed-form solution for it.
    \section{Performance}

    \section{Accuracy}

\chapter{Pricing Exotics}