\chapter{Implementation of the Methods}
    \section{General Problems and \texttt{mc\_price} Function}
        The error of discretization consists of two parts: the discretization error itself 
        (the one coming from the transition from an stochastic differential equation to the 
        stochastic difference equation) and the Monte-Carlo error (see Section \ref{Monte-Carlo:statistical}).
        We controlled the Monte-Carlo error with the following method:
        \begin{algorithm}
            \caption{Outer loop of the Monte-Carlo method (\texttt{mc\_price})}
            \label{alg:outer_loop}
            \begin{algorithmic}
                \State desired precision = 0.01 with confidence level 95\%
                \State prices = $\left[ \ \right]$
                \While{{len(prices\_confidence\_interval) > desired precision} \textbf{and} iter < MAX\_ITER}
                    \State paths = simulate()
                    \State prices $\gets$ payoff(paths)
                    \State prices\_confidence\_interval = confidence\_interval(prices)
                \EndWhile
                \State return mean(prices)
            \end{algorithmic}
        \end{algorithm}



    \section{Euler Scheme}

    \section{Andersen Scheme}

    \section{Broadie-Kaya Scheme}


\chapter{Comparison of the Methods}
    We shall compare the described methods for the European call option prices due to the fact that we have a closed-form solution for it.
    \section{Performance}

    \section{Accuracy}

\chapter{Pricing Exotics}