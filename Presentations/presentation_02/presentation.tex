\documentclass[aspectratio=169]{beamer}
\usetheme{vega}

\addbibresource{assets/pricing_lib.bib}

\DeclareMathOperator*{\plim}{\ensuremath{\operatorname{\P-lim}}}
\DeclareMathOperator*{\cor}{\ensuremath{\operatorname{cor}}}


\newcommand{\cA}{\mathcal{A}}
\newcommand{\cB}{\mathcal{B}}
\newcommand{\cC}{\mathcal{C}}
\newcommand{\cN}{\mathcal{N}}

\subtitle{Student Research Group 'Stochastic Volatility Models', Project 'Heston--2'}
\title{Monte-Carlo Simulation: Speed and Error Control}
\author{Artemy Sazonov, Danil Legenky, Kirill Korban}
\institute{Lomonosov Moscow State Univesity, Faculty of Mechanics and Mathematics}
\date{December 10, 2022}

\begin{document}
    \maketitle

    \section{Monte-Carlo Error Control}
        \begin{frame}{Central Limit Theorem}
            \begin{theorem}[Lindeberg-L\'evy]
                Let $X_1, \dots, X_n$ be a sequence of i.i.d. random variables with $\mathbb{E}[X_i] = \mu$ and $\var\left[X_i\right] = \sigma^2$. 
                Then as $n$ approaches infinity, the random variables $\sqrt{n}(\bar{X}_n - \mu)$ converge in law to a normal distribution $\cN(0, \sigma^2)$, i.e.
                \begin{equation}
                    \sqrt{n}\left(\bar{X}_n - \mu\right) \xrightarrow{d} \cN\left(0,\sigma^2\right).
                \end{equation}
            \end{theorem}
        \end{frame}
        \begin{frame}{Monte Carlo Simulation}{Statistical Estimation}
            \begin{lemma}
                Let $X_1, X_2, \dots, X_n$ be a series of independent and identically distributed random variables, and $h: \mathbb{R} \to \mathbb{R}$ be a borel function. Then $h(X_1), h(X_2), \dots, h(X_n)$ is a series of independent and identically distributed random variables.
            \end{lemma}
            Thus, we could write an unbiased consistent estimator of $\E \left[h(X)\right]$ as follows:
            \begin{equation}
                \widehat{\E \left[h(X)\right]} = \frac{1}{n} \sum_{i=1}^n h(X_i).
            \end{equation}
        \end{frame}

        \begin{frame}{Monte Carlo Simulation}{Local Truncation Error}
            Asymptotic confidence interval for $\hat{\mu} = \widehat{\E\left[X\right]}$ at the confidence level $\alpha$:
            \begin{equation}
                \mu \in \left(\hat{\mu} - z_{\alpha/2} \sqrt{\frac{\sigma^2}{n}}, \hat{\mu} + z_{\alpha/2} \sqrt{\frac{\sigma^2}{n}}\right).
            \end{equation}
            That means that the estimation error is equal to $2z_{\alpha/2} \sqrt{\frac{\sigma^2}{n}}$.
        \end{frame}

        \begin{frame}{Euler-Maruyama Discretization Scheme for SDEs}{Strong and weak convergence as a global truncation error analogue}
            \begin{definition}
                Let $\hat X^n(t)$ be a piecewise mesh approximation of an SDE solution $X(t)$ (we assume that there exists a unique strong solution). 
                Then a scheme is said to have a strong convergence of order $p$ if 
                \begin{equation}
                    \E\left[\left|\hat X^n(T) - X(T)\right|\right] \leq Ch^p, \quad n \to \infty.
                \end{equation}
                A scheme is said to have a weak convergence of order $p$ if for any polynomial $f: \R \to \R$ we have
                \begin{equation}
                    \left|\E\left[f(\hat X^n(T))\right] - \E\left[f(X(T))\right]\right| \leq Ch^p, \quad n \to \infty.
                \end{equation}
            \end{definition}
        \end{frame}

        \begin{frame}{Euler-Maruyama Discretization Scheme for SDEs}{Strong and weak convergence as a global truncation error analogue}
            \begin{theorem}
                Under some technical assumptions the Euler-Maruyama Discretization scheme \eqref{Euler:SDE} has a strong convergence of order $1/2$ and a weak convergence of order $1$.
            \end{theorem}
        
            \begin{nb}
                Since our goal is to approximate $\E\left[h(X)\right]$ with a given accuracy and the least possible number of simulations, we need to compare the weak convergence rate between the methods.
                But this will work only for the derivatives of the European type, since we cannot guarantee the convergence of the simulations for the times $t < T$.
            \end{nb}
        \end{frame}

    \section{Monte-Carlo Speed Control}
        \begin{frame}{Control Variate Method}{Motivation}
            \begin{itemize}
                \item Let us consider the following random variables: $X$ and $Y$.
                \item Let us suppose, that we somehow know $\E\left[X\right] = \mu_X$ and $\cor\left[X, Y\right] = \rho$.
                \item Our goal is to build a consistent estimator of $\E\left[Y\right] = \mu_Y$.
                \item Possible estimator is $\hat\mu_Y = \bar Y$. It is unbiased and consistent.
                \item Can we improve the speed of convergence of the estimator?
            \end{itemize}
        \end{frame}

        \begin{frame}{Control Variate Method}{Solution to the problem}
            Let us consider the following estimator:
            \begin{equation}
                \hat\mu_Y^b = \bar Y + b \left(\bar X - \mu_X\right).
            \end{equation}
            We can see that the estimator is consistent and unbiased. Let us minimise the variance of it using the following optimization problem:
            \begin{equation}
                \var\hat\mu_Y^b \to \min_{b \in \R}.
            \end{equation}
            The solution is to use $b^* = \sqrt{\frac{\var Y}{\var X}}\rho$.
        \end{frame}

        \begin{frame}{Other useful methods}{}
            \begin{itemize}
                \item Antithetic variates: $\frac{Y_1 + Y_2}{2}$, where $Y_1$ and $Y_2$ are the correlated series of iid random variables.
                \item Importance sampling: could be used to estimate the OTM options.
            \end{itemize}
        \end{frame}

    
    \section{Conclusion}
        \begin{frame}{To-dos}
            \begin{enumerate}
                \item How do we approximate the log-prices?
                \item Martingale correction in the Andersen schemes
                \item Numerical stability of implied volatility calculations
            \end{enumerate}
        \end{frame}

\end{document}
