\subsection{Truncated Gaussian Discretization Scheme}
    \begin{frame}{Truncated Gaussian Discretization Scheme}{Idea}
        \begin{block}{Andersen:}
            \emph{In this scheme the idea is to sample from a moment-matched Gaussian density where all probability
            mass below zero is inserted into a delta-function at the origin.}
        \end{block} 
        Same, but in the formular form:
        \begin{equation}
            \left(\left.\hat{V}(t+\Delta)\right| V(t)\right) = \left(\mu + \sigma Z\right)^+,
        \end{equation}
        where $Z$ is a standard normal random variable and $\mu$ and $\sigma$ are the 'mean' and the 'standard deviation' of the desired distribution.
        We find $\mu$ and $\sigma$ from the same old moment-matching techniques (see Slide \ref{frame:Andersen:denotemeanstd}).
    \end{frame}

    \begin{frame}{Truncated Gaussian Discretization Scheme}{Finding the constants}
        \begin{proposition}
            Let $\phi(x)$ be a standart Gaussian density and define a function $r:\mathbb{R} \to \mathbb{R}$ by the following equation:
            \begin{equation}
                r(x)\phi(r(x))+\Phi(r(x))(1+r(x)^2)= (1+x)\left(\phi(r(x)) + r(x)\Phi(r(x))\right)^2.
            \end{equation}
            Then the moment-matching parameters are
            \begin{align}
                \mu &= \frac{m}{\frac{\phi(r(\psi))}{r(\psi)} + \Phi(r(\psi))},\\ 
                \sigma &= \frac{m}{\phi(r(\psi)) + r(\psi)\Phi(r(\psi))}.
            \end{align}
        \end{proposition}
    \end{frame}

    \begin{frame}{Truncated Gaussian Discretization Scheme}{Finding the numerical integration interval}
        \textbf{Problem}: no closed-form solution for $r(\psi)$. 
        
        \textbf{Solution}: numerical solution.

        \textbf{Problem}: no known limits to use the numerical solution.

        \textbf{Solution}: 
        \begin{align}
            m   &= \frac{\delta^2}{2\beta} + \left(\hat{V}(t) - \frac{\delta^2}{2\beta}\right)e^{-2\beta \Delta},\\
            s^2 &= \frac{\hat{V}(t)\sigma^2e^{-2\beta \Delta}}{2\beta}\left(1 - e^{-2\beta \Delta}\right) + \frac{\delta^2\sigma^2}{8\beta^2}\left(1 - e^{-2\beta \Delta}\right)^2.
        \end{align}
    \end{frame}

    \begin{frame}{Truncated Gaussian Discretization Scheme}{Finding the numerical integration interval}
        \begin{equation}\label{eq:psi}
            \psi = \frac{s^2}{m^2} = \frac{\frac{\hat{V}(t)\sigma^2e^{-2\beta \Delta}}{2\beta}\left(1 - e^{-2\beta \Delta}\right) + \frac{\delta^2\sigma^2}{8\beta^2}\left(1 - e^{-2\beta \Delta}\right)^2}{(\frac{\delta^2}{2\beta} + \left(\hat{V}(t) - \frac{\delta^2}{2\beta}\right)e^{-2\beta \Delta})^2}.
        \end{equation}

        Differentiating this expression with respect to $V(t)$ shows that $\frac{\partial\psi}{\partial V(t)}<0$ for all $V(t)\geq 0$, such that the largest possible value for $\pi$ is 
        obtained for $V(t)=0$, and the smallest possible value for$V(t)=\infty$. Inserting these values for $V(t)$ into \eqref{eq:psi} shows that $\psi \in (0, \frac{\beta^2\sigma^2}{2\delta^2})$.
    \end{frame}

    \begin{frame}{Truncated Gaussian Discretization Scheme}{Finding the numerical integration interval}
        As a final computational trick, note that once we have established the function $r$ we can write
        
        \begin{equation}
        \mu = m\cdot f_\mu(\psi), \quad f_\mu(\psi)= \frac{r(\psi)}{\phi(r(\psi)) + r(\psi)\Phi(r(\psi))}
        \end{equation}

        \begin{equation}
        \sigma = s\cdot f_\sigma(\psi), \quad f_\sigma(\psi)= \frac{\psi^{-\frac{1}{2}}}{\phi(r(\psi)) + r(\psi)\Phi(r(\psi))}
        \end{equation}

        The two functions $f_\mu(\psi)$ and $f_\sigma(\psi)$ are ultimately what we should cache on a computer once and for all, on an equidistant grid for $\psi$ large enough to span the domain.
    \end{frame}
