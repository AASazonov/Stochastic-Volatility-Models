\subsection{About Andersen's Article}
\begin{frame}{Motivation}
    \begin{itemize}
        \item Euler scheme is not very accurate, but fast and easy to implement;
        \item Broadie-Kaya scheme is more accurate, but significantly slower and way more complicated;
    \end{itemize}
\end{frame}

\subsection{Quadratic-Exponential Discretization Scheme}
    \begin{frame}{Quadratic-Exponential Discretization Scheme}{}\label{frame:Andersen:denotemeanstd}
        We denote 
        \begin{align}
            m    &= \E\left[\left.\hat{V}(t+\Delta)\right| \hat{V}(t)\right], \\
            s^2  &= \E\left[\left.\left(\hat{V}(t+\Delta) - m\right)^2\right| \hat{V}(t)\right], \\
            \psi &= \frac{s^2}{m^2}.
        \end{align}
    \end{frame}

    \begin{frame}{Quadratic-Exponential Discretization Scheme}{Idea}
        Andersen proposes an approximation based on moment-matching techniques. His goal is then to speed up the first step of Broadie and Kaya's method.
        He observes that the conditional distribution of $\hat{V}(t+\Delta)$ given $\hat{V}(t)$ visually difers when $\hat{V}(t)$ is small or large (in the variation coefficient sense).
        The scheme is constructed from the following two subschemes:
        \begin{enumerate}
            \item Quadratic sampling scheme ($\psi \leq 2$);
            \item Exponential sampling scheme ($\psi \geq 1$).
        \end{enumerate}
        Fortunately, these two intervals cover the whole positive real line. Furthermore, these two schemes could be applied at the same time when $\psi\in[1, 2]$. This implicates that there exist some critical value $\psi_{\text{crit}}\in[1, 2]$, which could be an indicator of which scheme is more applicable at the given value of $\psi$. Let us show you this.
    \end{frame}

    \begin{frame}{Quadratic-Exponential Discretization Scheme}{The Problem with Variance}
        For large enough $\hat{V}(t)$ (in the $CV$-sense) we can approximate the distribution of $\hat{V}(t+\Delta)$ by the scaled non-central chi-squared distribution with $1$ degree of freedom:
        \begin{align}
            \law\left(\left.\hat{V}(t+\Delta) \right| \hat V(t)\right) =  a(\Delta, \hat{V}(t), VP) \chi'^2_1(b(\Delta, \hat{V}(t), VP)),
        \end{align}
        where $VP$ is the vector of parameters of the CIR variance.
        However, if $\hat{V}(t)$ is close to zero, then we have a problem in finding such $a = a(\Delta, \hat{V}(t), VP)$ and $b = b(\Delta, \hat{V}(t), VP)$ such that the moments of the desired conditional distribution could be properly matched.
    \end{frame}

    \begin{frame}{Quadratic-Exponential Discretization Scheme}{The Problem with Variance}
        Therefore, we approximate the desired distribution with the following method. Let $\xi$ and $\eta$ be independent random variables and  $\xi \sim Be(1-p)$, $\eta \sim Exp(\beta)$ for some $p \in (0, 1)$ and $\beta > 0$. Then we have (given $\hat{V}(t)$)
        \begin{equation}
            \hat{V}(t+\Delta) = \xi\cdot\eta,
        \end{equation}
        what gives us the following distribution density:
        \begin{equation}
            p_{\hat{V}(t+\Delta)\vert \hat{V}(t)} = p\cdot \delta(x) + (1-p) \cdot\beta e^{-\beta x},
        \end{equation}
        where $\delta(x)$ is a standart delta function and for some $\beta$ and $p$.
        Sampling $\xi$ and $\eta$: Smirnov's transform. Or we can use the Smirnov transform with the cdf of the desired distribution.
    \end{frame}

    \begin{frame}{Quadratic-Exponential Discretization Scheme}{Finding the constants}
        \begin{lemma}
            We have
            \begin{align}
                b^2 &= \frac{2}{\psi} -1 +\sqrt{\frac{2}{\psi}\left(\frac{2}{\psi}-1\right)}, \\ 
                a   &= \frac{m}{1+b^2}.
            \end{align}
        \end{lemma}
        \begin{proof}
            Plain equating of the theoretical and real moments.
        \end{proof}
        \textbf{Remark:} The above lemma is not valid for $\psi \geq 2$.
    \end{frame}

    \begin{frame}{Quadratic-Exponential Discretization Scheme}{Finding the constants}
        \begin{lemma}
            We have
            \begin{equation}
                p     = \frac{\psi - 1}{\psi + 1}, \qquad \beta = \frac{1-p}{m} = \frac{2}{m(\psi+1)}.
            \end{equation}
        \end{lemma}
        \begin{proof}
            By direct integration of the given densities we get the following:
            \begin{equation}
                \frac{1-p}{\beta} = m, \qquad \frac{1-p^2}{\beta^2} = s^2.
            \end{equation}
        \end{proof}
        \textbf{Remark:} The above lemma is not valid for $\psi \leq 1$.
    \end{frame}


\subsection{Truncated Gaussian Discretization Scheme}
    \begin{frame}{Truncated Gaussian Discretization Scheme}{Idea}
        \begin{block}{Andersen:}
            \emph{In this scheme the idea is to sample from a moment-matched Gaussian density where all probability
            mass below zero is inserted into a delta-function at the origin.}
        \end{block} 
        Same, but in the formular form:
        \begin{equation}
            \left(\left.\hat{V}(t+\Delta)\right| V(t)\right) = \left(\mu + \sigma Z\right)^+,
        \end{equation}
        where $Z$ is a standard normal random variable and $\mu$ and $\sigma$ are the 'mean' and the 'standard deviation' of the desired distribution.
        We find $\mu$ and $\sigma$ from the same old moment-matching techniques (see Slide \ref{frame:Andersen:denotemeanstd}).
    \end{frame}

    \begin{frame}{Truncated Gaussian Discretization Scheme}{Finding the constants}
        \begin{proposition}
            Let $\phi(x)$ be a standart Gaussian density and define a function $r:\mathbb{R} \to \mathbb{R}$ by the following equation:
            \begin{equation}
                r(x)\phi(r(x))+\Phi(r(x))(1+r(x)^2)= (1+x)\left(\phi(r(x)) + r(x)\Phi(r(x))\right)^2.
            \end{equation}
            Then the moment-matching parameters are
            \begin{align}
                \mu &= \frac{m}{\frac{\phi(r(\psi))}{r(\psi)} + \Phi(r(\psi))},\\ 
                \sigma &= \frac{m}{\phi(r(\psi)) + r(\psi)\Phi(r(\psi))}.
            \end{align}
        \end{proposition}
    \end{frame}

    \begin{frame}{Truncated Gaussian Discretization Scheme}{FInding the numerical integration interval}
        \textbf{Problem}: no closed-form solution for $r(\psi)$. 
        
        \textbf{Solution}: numerical solution.

        \textbf{Problem}: no known limits to use the numerical solution.

        \textbf{Solution}: 
        % kappa = 2\beta, theta = \frac{\delta^2}{2\beta}
        \begin{align}
            m   &= \frac{\delta^2}{2\beta} + \left(\hat{V}(t) - \frac{\delta^2}{2\beta}\right)e^{-2\beta \Delta},\\
            s^2 &= \frac{\hat{V}(t)\sigma^2e^{-2\beta \Delta}}{2\beta}\left(1 - e^{-2\beta \Delta}\right) + \frac{\delta^2\sigma^2}{8\beta^2}\left(1 - e^{-2\beta \Delta}\right)^2.
        \end{align}
        Then we analyze $\psi$ wrt $\hat{V}(t)$ and obtain a finite interval as a domain for $r(\psi)$.
    \end{frame}