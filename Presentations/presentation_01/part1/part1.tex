\subsection{Stochastic Volatility Model}

    \begin{frame}
        \onslide<1->
            Assume that the spot asset at time $t$ follows the diffusion
            \begin{equation}\label{Heston93:eq0}
                dS(t) = \mu S(t)dt + \sqrt{v(t)} S(t) dZ_1(t)
            \end{equation}
        
        \onslide<2->
            If the volatility follows an Ornstein-Uhlenbeck process, 
            \begin{equation}
                d\sqrt{v(t)} = -\beta \sqrt{v(t)}dt +\delta dZ_2(t),
            \end{equation}
            then It\^{o}'s lemma shows that the variance $v(t)$ follows the process
        
        \onslide<3->
            \begin{equation}
                dv(t) = \left(\delta^2 - 2\beta v(t)\right) dt + 2\delta \sqrt{v(t)} dZ_2(t),
            \end{equation}
            where $Z_1$, $Z_2$ are the Wiener processes, and $\correlation\left(Z_1, Z_2\right) = \rho$
    \end{frame}

    \begin{frame}
        \onslide<1->
            For  simplicity  at  this  stage, we  assume  a  constant  interest  rate  $r$.  
            Therefore,  the  price  at  time  $t$  of a  unit  discount  bond  that  matures  at  time $t+\tau$ is
            \begin{equation}
                P(t, t + \tau) = e^{-r\tau}.
            \end{equation}
            
        \onslide<2->
            These assumptions are still insufficient to price some instruments because we have not yet made an 
            assumption that gives us the 'price of volatility risk'. Standart arbitrage arguments demonstrate that the value
            of any asset $U(S, v ,t)$ must satisfy the following PDE:
            \begin{multline}\label{Heston93:assetpde}
                \frac{1}{2}vS^2\frac{\partial^2 U}{\partial S^2} + \rho\sigma v S \frac{\partial^2 U}{\partial S \partial v} + \frac{1}{2}\sigma^2 v \frac{\partial^2 U}{\partial v^2} + 
                rS\frac{\partial U}{\partial S} +\\+ \left(\kappa(\theta - v(t)) - \lambda(S, v, t)\right)\frac{\partial U}{\partial v}
                - rU + \frac{\partial U}{\partial t} = 0
            \end{multline}
    \end{frame}

    \begin{frame}
        \onslide<1->
            \begin{multline}
                \frac{1}{2}vS^2\frac{\partial^2 U}{\partial S^2} + \rho\sigma v S \frac{\partial^2 U}{\partial S \partial v} + \frac{1}{2}\sigma^2 v \frac{\partial^2 U}{\partial v^2} + 
                rS\frac{\partial U}{\partial S} +\\+ \left(\kappa(\theta - v(t)) - \lambda(S, v, t)\right)\frac{\partial U}{\partial v}
                - rU + \frac{\partial U}{\partial t} = 0 \nn
            \end{multline}
            The unspecified term $\lambda(S, v, t)$ represents the price of volatility risk and \textbf{must} be independent of the particular asset.        
        
        \onslide<2-> 
            To motivate the choice of $\lambda(S, v, t)$, we note that in Breeden's consumption-based model,
            \begin{equation}
                \lambda(S, v, t)dt = \gamma \covariance \left(dv, \frac{dC}{C}\right),
            \end{equation}
            where $C(t)$ is the consumption rate, and $\gamma$ is the relative-risk aversion of an investor. 
    \end{frame}

    \begin{frame}
        \onslide<1->
            Consider the consumption process $C(t)$
            \begin{equation}
                dC(t) = \mu_c v(t) C(t) dt + \sigma_C \sqrt{v(t)} CdZ^3(t).
            \end{equation}
            where consumption growth has constant correlation with the spot-asset return.
            This generates a risk premium proportional to the variance process. 
    \end{frame}

    \begin{frame}
        A European  call  option  with  strike   $K$  and  maturity $T$
        satisfies  the  PDE  \eqref{Heston93:assetpde}  subject  to  the  following  boundary  conditions:
        \begin{align}
            & U(S, v, t) = \max \left(0, S-K\right) \\
            & U(0, v, t) = 0 \\
            & \frac{\partial U}{\partial S}(\infty, v, t) = 1 \\
            & rS \frac{\partial U}{\partial S}(S, 0, t) + \kappa \theta\frac{\partial U}{\partial v}(S, 0, t) - r U(S, 0, t) +\frac{\partial U}{\partial t}(S, 0, t) = 0 \\
            & U(S, \infty, t) = S.
        \end{align}
    \end{frame}

    \begin{frame}
        \onslide<1->
        By analogy with the BS formula, we guess a solution of the form
        \begin{equation}
            C(S, v, t) = S P_1 - KP(t, T)P_2,
        \end{equation}
        where the first term is the present value of the spot asset upon optimal exercise, and 
        the second term is the PV of the strike-price payment. Both of these terms 
        \textbf{must satisfy} the PDE \eqref{Heston93:assetpde}. 
        \onslide<2->
        We shall get the desired probabilities via substituting 
        the price $C$ and probabilities' characteristic functions\footnote{in the Appendix Heston shows that those satisfy our PDE too with some other boundary conditions} in the \eqref{Heston93:assetpde} and using the inverse 
        formula to calculate the explicit value, which is equal to
        \begin{align}
            & P_j(\log S, v, T; \log K) = \frac{1}{2} + \frac{1}{\pi}\int_{0}^{\infty}\Re\left(\frac{e^{-i\phi\log K} f_j(\log S, v, T; \phi)}{i\phi}\right) d\phi \\ 
            & f_j(\log S, v, T; ) = e^{C(T-t, \phi) + D(T-t, \phi)v +i\phi\log S} \text{ for some known C, D}
        \end{align}
    \end{frame}

\subsection{Bond Options, Currency Options, and Other Extensions}

    \begin{frame}
        \onslide<1->
        One  can  incorporate  stochastic  interest  rates  into  the  option  pricing model,
        following  Merton  (1973)  and  Ingersoll  (1990).  In  this  manner,
        one  can  apply  the  model  to  options  on  bonds  or  on  foreign  
        currency. This  section  outlines  these  generalizations  to  show  the  
        broad  applicability  of  the  stochastic  volatility  model.  
        
        \onslide<2->These generalizations  are equivalent  to  the  model  of  the  previous  section,  
        except  that  certain parameters  become  time-dependent  to  reflect  the  
        changing  characteristics  of  bonds  as  they  approach  maturity. To  
        incorporate  stochastic  interest  rates,  we  modify  equation  
        \eqref{Heston93:eq0}  to allow  time  dependence  in  the  volatility  of  the  
        spot  asset:
        \begin{equation}\label{Heston93:stspot}
            dS(t) = \mu_S S(t) dt + \sigma_S(t) \sqrt{(v(t))}S(t)  dZ_1(t)
        \end{equation}
    \end{frame}

    \begin{frame}
        Although  the  results  of  this  section  do  not  depend  on  the specific
        form  of $\sigma_S$,  if  the  spot  asset  is  a  discount  bond  then  
        $\sigma_S$  must vanish  at  maturity  in  order  for  the  bond  price  
        to  reach  par  with  probability  $1$.  The  specification  of  the  drift
        term  $\mu_S$ is  not important  because it  will  not  affect  option  prices. 
        We  specify  analogous  dynamics  for the  bond  price:
        \begin{equation}\label{Heston93:stbond}
            dP(t; T) = \mu_P P(t; T) dt + \sigma_P(t) \sqrt{(v(t))} P(t; T)dZ_2(t)
        \end{equation}
    \end{frame}
    
    \begin{frame}
        we assume that the variances of both the spot asset and the bond are determined 
        by the same variable $v(t)$. In this model, the valuation equation is

        \begin{multline}
            \frac{1}{2}\sigma_S(t)^2vS^2\frac{\partial^2U}{\partial S^2} + \frac{1}{2}\sigma_P^2(t)vP^2\frac{\partial^2U}{\partial P^2} + \frac{1}{2}\sigma^2v\frac{\partial^2U}{\partial v^2} +\\+ \rho_{SP}\sigma_S(t)\sigma_P(t)vSP\frac{\partial^2 U}{\partial S \partial P} +
            \rho_{Sv}\sigma_S(t)\sigma vS\frac{\partial^2 U}{\partial S\partial v} +\\+ \rho_{Pv}\sigma_P(t)\sigma vP\frac{\partial^2 U}{\partial P \partial v} + rS\frac{\partial U}{\partial S} + rP\frac{\partial U}{\partial P} + \\ +
            \left(\kappa(\theta - v(t))-\lambda v\right)\frac{\partial U}{\partial v} - rU +\frac{\partial U}{\partial t} = 0
        \end{multline}
    \end{frame}

    \begin{frame}
        One  can  also  apply  the  model  when  the  spot  asset  $S(t)$  is  
        the  dollar price  of  foreign  currency.  We  assume  that  the  foreign 
        price  of  a  foreign discount  bond,  $F(t;  T)$,  follows  
        dynamics  analogous  to  the  domestic bond  in  equation \eqref{Heston93:stbond}:
        \begin{equation}\label{Heston93:stfx}
            dF(t; T) = \mu_P F(t; T) dt + \sigma_P(t) \sqrt{(v(t))} F(t; T)dZ_2(t)
        \end{equation}

        For clarity, we denote the domestic interest rate by $r_D$, and the foreign one by $r_F$.
    \end{frame}

    \begin{frame}
        Although  the  stochastic  interest  rate  models  of  this  section  are
        tractable,  they  would  be  more  complicated  to  estimate  than  the  
        simpler  model  of  the  previous  section.  For  short-maturity  options  
        on equities,  any  increase  in  accuracy  would  likely  be  outweighed  by  the
        estimation  error  introduced  by  implementing  a  more  complicated
        model.  As  option  maturities  extend  beyond  one  year,  however,  the
        interest  rate  effects  can  become  more  important  [Koch  (1992)].  The
        more  complicated  models  illustrate  how  the  stochastic  volatility  model
        can  be  adapted  to  a  variety  of  applications.
    \end{frame}

\subsection{Effects  of  the  Stochastic  Volatility  Model  Options  Prices}

    \begin{frame}
        \onslide<1->
        In  this  section,  Heston  studies  the  effects  of  stochastic  volatility  
        on  options prices  and  contrasts  results  with  the traditional Black-Scholes  model.  
        Many  effects are  related  to  the  time-series  dynamics  of  volatility.  
        For  example,  a higher  variance  $v(t)$  raises  the  prices  of  all  options,  
        just  as  it  does  in the   Black-Scholes   model.   
        \onslide<2->
        In   the   risk-neutralized   pricing   probabilities,  the  variance  
        follows  a  square-root  process
        \begin{equation}
            dv(t) = \kappa^* \left(\theta^* - v(t)\right)dt + \sigma \sqrt{v(t)}dZ_2(t),
        \end{equation}
        where $\kappa^* = \kappa + \lambda$, and $\theta^* = \frac{\kappa\theta}{\kappa+\theta}$.
        The stochastic volatility model can conveniently explain properties of option prices in 
        terms of the underlying distribution of spot returns. To illustrate effects on options 
        prices, we shall use the parameters in Table 1.
    \end{frame}

    \begin{frame}
        \begin{figure}
            \includegraphics[width=\linewidth]{assets/part1/fig01.png}
        \end{figure}
    \end{frame}

    \begin{frame}{Probability Density}
        \begin{figure}
            \includegraphics[height=180pt]{assets/part1/fig02.png}
        \end{figure}
    \end{frame}
    
    \begin{frame}{Price Difference (\$)}
        \begin{figure}
            \includegraphics[height=180pt]{assets/part1/fig03.png}
        \end{figure}
    \end{frame}

    \begin{frame}
        The parameter $\sigma$ controls the volatility of volatility. When $\sigma$ is zero, the volatility is deterministic, and continuously compounded spot returns have a normal distribution. Otherwise, $\sigma$ increases the kurtosis of spot returns. Figure 3 shows how this creates two fat tails in the distribution of spot returns. 
        As Figure 4 shows, this has the effect of raising far-in-the-money and far-out-of-the-money option prices and lowering near-the-money prices. Note, however, that there is little effect on skewness or on the overall pricing of in-the-money options relative to out-of-the-money options.
    \end{frame}

    \begin{frame}{Probability Density}
        \begin{figure}
            \includegraphics[height=180pt]{assets/part1/fig04.png}
        \end{figure}
    \end{frame}

    \begin{frame}{Price Difference (\$)}
        \begin{figure}
            \includegraphics[height=180pt]{assets/part1/fig05.png}
        \end{figure}
    \end{frame}