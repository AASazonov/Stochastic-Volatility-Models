\begin{frame}{Broadie-Kaya Simulatation Method}{Definition}
    It folows from Heston model, that for $t > u$
    \begin{align}
         S_t &= S_u e^{\left( r(t-u)-\frac{1}{2} \int_{u}^{t} v_s \, ds  + \rho\int_{u}^{t} \sqrt{v_s} \, dZ_1(s) + (1-\rho)\int_{u}^{t} \sqrt{v_s} \, dZ_2(s)  \right)}, \label{BK:Price_int} \\
         v_t &= v_u + \kappa\theta(t-u) - \kappa \int_{u}^{t} v_s \, ds + \sigma\int_{u}^{t} \sqrt{v_s} \, dZ_2(s), \label{BK:Vol_int}
    \end{align}
    Exact Simulation Algorithm for the SV Model
    \begin{enumerate}
        \item[Step 1] Generate a sample from the distribution of $v_t$ given $v_u$.
        \item[Step 2] Generate a sample from the distribution of $\int_{u}^t V_s ds$ given $v_t$ and $v_u$.
        \item[Step 3] Recover $\int_{u}^t \sqrt{v_s} dZ_1(s)$ given $v_t$, $v_u$, and
        $\int_{t}^u v_s ds$
        \item[Step 4] Generate a sample from the distribution of $S_t$ given $\int_{u}^t \sqrt{v_s} dZ_1(s)$ , $\int_{u}^t \sqrt{v_s} dZ_2(s)$ and $\int_{u}^t v_s ds$
    \end{enumerate}
\end{frame}

\begin{frame}{Broadie-Kaya Simulatation Method}{Step 1: Generate a sample from the distribution of $v_t$ given $v_u$}
    As shown in Cox et al. (1985) the distribution of $v_t$ given $v_u$ for some u < t is, up to a  scale factor, a noncentral chi-squared distribution. The transition law of $v_t$ can be expressed as:

    \begin{equation}
        v_t = \frac{\sigma^2(1-e^{-\kappa(t-u)})}{4\kappa}\chi_d'^{2}\left(\frac{4\kappa e^{-\kappa(t-u)}}{\sigma^2(1-e^{-\kappa()t-u})} v_u  \right), t > u , \label{BK:vol_law}
    \end{equation}

    where $\chi_d'^{2}(\lambda)$ denotes the noncentral chi-squared random
    variable with d degrees of freedom, and noncentrality
    parameter $\lambda$, and

    \begin{equation}
        d = \frac{4\theta\kappa}{\sigma^2} \label{BK:vol_law:parameter}
    \end{equation}
    
\end{frame}


\begin{frame}{Broadie-Kaya Simulatation Method}{Step 1: Generate a sample from the distribution of $v_t$ given $v_u$}
    Thus, we can sample from the distribution of $v_t$ exactly,
    provided that we can sample from the noncentral chisquared distribution.

    Johnson et al. (1994) show that for $d > 1$, the following representation is valid:


    \begin{equation}
        \chi_d'^{2}(\lambda) = \chi_1'^{2}(\lambda) + \chi_{d-1}'^{2} = N(\lambda, 1)^2 + \chi_{d-1}^{2}
    \end{equation}

    Therefore,  when $d > 1$ sampling from a noncentral chi-squared distribution
    is reduced to sampling from an ordinary chi-squared and
    an independent normal.

    When $d < 1$ we can use the the fact that:

    \begin{equation}
        \chi_d'^{2}(\lambda) \sim \chi_{d + 2N}^{2}
    \end{equation}

    Where $N$ is a Poisson random variable with mean $\frac{\lambda}{2}$

    
\end{frame}

\begin{frame}{Broadie-Kaya Simulatation Method}{Step 2:  Generate a sample from the distribution of $\int_{u}^t V_s ds$ given $v_t$ and $v_u$}
    The folowing formula can be derived. The dirivation ypu can see in in the original paper
    \begin{multline}
        \phi(a) = \E\left[  \exp{(ia\int_{u}^t V_s ds)} \Bigg| v_u, v_t\right] = \frac{\gamma(a)e^{-(1/2)(\gamma(a) - \kappa)(t - u)}}{\kappa(1-e^{-\gamma(a)(t-u)})} \\
        \exp{\left(\frac{v_u + v_t}{\sigma^2} \left[  \frac{\kappa(1+e^{-\kappa(t-u)})}{1-e^{-\kappa(1-u)}}                 \right]     \right)}
        \frac{I_{0.5d-1}(\sqrt{v_uv_t}\frac{4\gamma(a)e^{-0.5\gamma(a)(t-u)}}{\sigma^2(1-e^{-\kappa(a)(t-u)})})}{I_{0.5d-1}(\sqrt{v_uv_t}\frac{4\kappa e^{-0.5\kappa(t-u)}}{\sigma^2(1-e^{-\kappa(t-u)})})}
    \end{multline}
    Where $\gamma(a) = \sqrt{\kappa^2 - 2\sigma i a}$ and $I_{0.5d-1}$ is a modified  Bessel function of the first kind

\end{frame}



\begin{frame}{Broadie-Kaya Simulatation Method}{Step 2:  Generate a sample from the distribution of $\int_{u}^t V_s ds$ given $v_t$ and $v_u$}
    
    Let $V(u,t)$  denote the random
    variable that has the same distribution as the integral $\int_{u}^t V_s ds$ conditional on $v_u$ and $v_t$
    
    Then we need to invert the characteristic function to get Cumulative distribution function
    
    \begin{multline}
        F(x) = P(V(u, t) \leq x) = E\left[ e^{iaV(u,t)} \Big| v_u, v_t\right] = \frac{1}{\pi} \int_{-\infty}^\infty \frac{\sin ux}{u} \Phi(u) du \\
        = \frac{2}{\pi} \int_{0}^\infty \frac{\sin ux}{u} \Phi(u) du
    \end{multline}
    
    
\end{frame}

\begin{frame}{Broadie-Kaya Simulatation Method}{Step 2:  Generate a sample from the distribution of $\int_{u}^t V_s ds$ given $v_t$ and $v_u$}

    To calculate the integral trapezoidal rule is used, because the errors tend to cancel  for periodic and other oscillating integrands

    \begin{equation}
        P(V(u, t) \leq X) = \frac{hx}{\pi} + \frac{2}{\pi} \sum_{j=1}^\infty \frac{\sin hjx}{j} Re[\Phi(hj)] - e_d(h)
    \end{equation}
    Where h is a grid size and $e_d(h)$ is the discretization   error

\end{frame}

\begin{frame}{Broadie-Kaya Simulatation Method}{Step 2:  Generate a sample from the distribution of $\int_{u}^t V_s ds$ given $v_t$ and $v_u$}

    The discretization error $e_d$ can be bounded above and
below by using a Poisson summation formula 

    \begin{equation}
        0 \leq e_d(h) = \sum_{k=1}^\infty[ F(\frac{2k\pi}{h} + x) - F(\frac{2k\pi}{h} - x)] \leq 1 - F(\frac{2\pi}{h} - x)
    \end{equation}

    If we want to achieve a discretization error $\alpha$, then the
    step size should be

    \begin{equation}
        h = 2\frac{2\pi}{x+ u_\alpha} \geq \frac{\pi}{u_\alpha}
    \end{equation}

    where $1-F(u_\alpha) = \alpha$ and $0 \leq x \leq u_\alpha$


\end{frame}



\begin{frame}{Broadie-Kaya Simulatation Method}{Step 2:  Generate a sample from the distribution of $\int_{u}^t V_s ds$ given $v_t$ and $v_u$}


    To be able to calculate $P(V(u, t) < x )$ using (28), we
    need to determine a point at which the summation can be
    terminated. Let N represent the last term to be calculated
    so that the approximation becomes

     \begin{equation}
        F(x) = P(V(u, t) \leq X) = \frac{hx}{\pi} + \frac{2}{\pi} \sum_{j=1}^N \frac{\sin hjx}{j} Re[\Phi(hj)] - e_d(h) - e_T(N)
    \end{equation}

    Because $|\sin(ux)| \leq 1$, the integrand in (29) is bounded by
     
    \begin{equation}
        \frac{2|RE[\Phi(u)]|}{\pi u} \leq \frac{2[\Phi(u)]}{\pi u}
    \end{equation}

    

\end{frame}

\begin{frame}{Broadie-Kaya Simulatation Method}{Step 2:  Generate a sample from the distribution of $\int_{u}^t V_s ds$ given $v_t$ and $v_u$}
    
    Because the integrand is oscillating, the bound for the last term gives a good estimate for the truncation error, i.e. we cam set $e_t(N) = h \frac{2[\Phi(u)]}{\pi u} $ The summation is terminated at j = N when

    \begin{equation}
        \frac{2[\Phi(u)]}{\pi u} < e_t(N)/x
     \end{equation}
    
\end{frame}



\begin{frame}{Broadie-Kaya Simulatation Method}{Step 2:  Generate a sample from the distribution of $\int_{u}^t V_s ds$ given $v_t$ and $v_u$}

    
    To simulate the value of the integral, the inverse transform method is used. We generate a uniform random variable U and then find the value of x for which 

    \begin{equation}
        P(V(u, t) \leq x) = U
    \end{equation}

\end{frame}


\begin{frame}{Broadie-Kaya Simulatation Method}{Step 3: Recover $\int_{u}^t \sqrt{v_s} dZ_1(s)$ given $v_t$, $v_u$, and
        $\int_{t}^u v_s ds$}

    The following formula can be used to calculate $\int_{u}^t \sqrt{v_s} dZ_1(s)$, as we already generated samples for $v_t , v_u, V(u, t)$


    \begin{equation}
        \int_{u}^t \sqrt{v_s} dZ_1(s) = \frac{1}{\sigma}(v_t v_u) - \kappa\theta(t-u) + V(u, t)
    \end{equation}
    
    

    
\end{frame}





\begin{frame}{Broadie-Kaya Simulatation Method}{Step 4 Generate a sample from the distribution of $S_t$ given $\int_{u}^t \sqrt{v_s} dZ_1(s)$ , $\int_{u}^t \sqrt{v_s} dZ_2(s)$ and $\int_{u}^t v_s ds$}

    Lastly we need to bring everything together:\\
    $\int_{u}^t \sqrt{v_s} dZ_1(s)$ , $\int_{u}^t \sqrt{v_s} dZ_2(s)$ are already calculated \\
    $V(u, t) = \int_{u}^t v_s ds $ also calculated

    \begin{equation}
        S_t = S_u \exp{\left( r(t-u)-\frac{1}{2} V(u, t)  + \rho\int_{u}^{t} \sqrt{v_s} \, dZ_1(s) + (1-\rho)\int_{u}^{t} \sqrt{v_s} \, dZ_2(s)  \right)}
    \end{equation}

        
\end{frame}


